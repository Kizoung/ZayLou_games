\documentclass{article}                             % Classe du document: article

\usepackage[french]{babel}                          % Langue du document: français
\usepackage[utf8]{inputenc}                         % Encodage du document: UTF-8
\usepackage[T1]{fontenc}                            % Encodage des polices: T1 
\usepackage{fancybox}                               % Boîtes de texte stylisées
\usepackage{hyperref}                               % Liens hypertexte
\usepackage{color}                                  % Couleurs
\usepackage{graphicx}                               % Images
\usepackage{enumitem}                              % List customization

\title{IFT3150 A-E25: Description de projet \\      % Titre du document
        Nom du projet: ZayLou Games}                % Titre du document

\author{                                            % Auteurs du document
    Lallia Diakité: 20256054 \\
    Marc Olivier Jean Paul: 20241763
}                 
\date{\today}                                       % Date du document   

\begin{document}                                    % Début du document

\maketitle                                          % Afficher le titre

\noindent{Lien vers le dépôt GitHub: \url{https://github.com/Kizoung/ZayLou_games.git}}


\section{Introduction}                              % Section 1: Introduction
\label{sec:intro}                                   % Étiquette de la section
Ce document présente le projet \textbf{ZayLou Games} développé dans le cadre du cours IFT3150 à l'été 2025. 
Ce projet est supervisé par Louis-Édouard Lafontant et réalisé par Lallia Diakité et Marc Olivier Jean Paul. 
Il vise à concevoir une plateforme innovante de création de jeux vidéo interactifs.

\section{Contexte}                                   % Section 2: Contexte
\label{sec:contexte}                                % Étiquette de la section
La création de jeux vidéo est aujourd’hui facilitée par des plateformes telles
que \textbf{Scratch}, \textbf{Unity Playground}, ou \textbf{Gamefroot}. 
Celles-ci proposent des interfaces simplifiées, souvent basées sur le "glisser-déposer" 
(\textit{drag-and-drop}), permettant aux jeunes ou aux novices de concevoir des jeux 
sans écrire de code. Cependant, ces outils restent parfois limités dans la 
personnalisation ou la profondeur des mécaniques de jeu qu’ils permettent de développer.

Parallèlement, la technologie \textbf{NFC (Near Field Communication)} gagne en popularité dans le domaine du jeu vidéo. On la retrouve notamment dans les figurines \textit{Amiibo de Nintendo} qui permettent de débloquer du contenu dans certains jeux, ou encore dans des cartes interactives utilisées dans les \textit{escape games} ou les jeux éducatifs. Cette technologie permet d’ajouter une dimension physique et tangible à l’expérience vidéoludique.

\section{Problématiques et motivations}     % Section 3: Problématiques et motivations
\label{sec:problématiques}                        % Étiquette de la section

Actuellement, il existe de nombreuses plateformes de création de jeux-vidéos,
Mais elles présentent plusieurs problèmatiques:

\begin{itemize}
  \item \textbf{Complexité excessive} pour les jeunes utilisateurs : des outils comme Unity 
  ou Construct, bien que puissants, peuvent décourager les débutant par leur interface technique.
  \item \textbf{Peu d’interactivité physique :} Peu de plateformes permettent l'intégration de composants
    physiques comme les cartes NFC, réservées aux jeux commerciaux complexes.
  \item \textbf{Accessibilité limitée :} Certaines plateformes ne sont pas compatibles avec les appareils mobiles ou 
  nécessitent des installations complexes.
\end{itemize}

Nous cherchons donc à répondre à ces enjeux en proposant une solution :
\begin{itemize}
  \item \textbf{Accessible et intuitive}, pour permettre aux jeunes de concevoir leur jeux sans 
  connaissances préalables;
  \item \textbf{Originale} grâce aux cartes NFC pour enrichir l'expérience de jeu;
  \item Favorisant la créativité des jeunes.
\end{itemize}


\section{Proposition et objectifs}
\label{sec:proposition}                            % Étiquette de la section
Nous proposons de créer \textbf{ZayLou Games} une plateforme de création de jeux vidéo, pour
laquelle nous aurons à :
\begin{itemize}[leftmargin=1.5em]
  \item Créer une plateforme plateforme web pour la création de jeux 2D accessible aux non-développeurs grâce à une interface drag-and-drop.
  \item Créer une application mobile pour l'exécution des jeux.
  \item Intégrer l’utilisation de cartes NFC permettant de déclencher des effets en jeu.
  \item Simuler un effet de profondeur, sans recourir à la manipulation directe de modèles 3D.
  \item Nous assurer que le jeu créé peut être sauvegardé et joué sur une application mobile compatible.
\end{itemize}

\subsection{Cartes et puces NFC}
\begin{itemize}[leftmargin=1.5em]
  \item Les utilisateurs peuvent scanner des cartes NFC pour déclencher des effets dans le jeu (bonus, malus, surprises).
  \item Les cartes contiennent des puces NFC (Near Field Communication).
  \item Chaque puce stocke un fichier JSON décrivant les effets associés.
  \item Ces fichiers sont lus par la plateforme pour appliquer les effets en jeu.
\end{itemize}

\subsection{Langages de programmation}
Le développement sera divisé entre :
    \begin{itemize}[leftmargin=1.5em]
      \item \textbf{Front-end :}
        \begin{itemize}[leftmargin=1.5em]
          \item Utilisation de React Native, une bibliothèque JavaScript pour créer des interfaces mobiles et web.
          \item Une seule base de code pour plusieurs plateformes.
          \item Création d'une interface intuitive et réutilisable grâce aux composants.
        \end{itemize}

      \item \textbf{Back-end :}
        \begin{itemize}[leftmargin=1.5em]
          \item Utilisation de Node.js, environnement d’exécution JavaScript côté serveur.
          \item Parfait pour les applications en temps réel.
          \item Intégration facile avec MongoDB (base NoSQL) pour la gestion des données utilisateurs et des jeux.
        \end{itemize}

      \item \textbf{WebSockets:}
        \begin{itemize}[leftmargin=1.5em]
          \item Gérer la communication en temps réel entre le front-end et le back-end.
          \item Permettre aux utilisateurs de jouer en ligne et d'interagir avec les jeux créés.
          \item Offrir une expérience fluide et réactive.
        \end{itemize}
    \end{itemize}


\section{Méthodologie}                              % Section 5: Méthodologie de développement
\label{sec:methodologie}                                    % Étiquette de la section
Nous utiliserons la méthodologie Agile pour le développement de la plateforme ZayLou
Games. Il s'agit d'une approche de développement logiciel qui met l'accent sur
la collaboration entre les membres de l'équipe, la flexibilité et l'adaptabilité aux
changements. Elle est particulièrement adaptée pour un projet de la taille de ZayLou Games.

Chaque semaine, nous organiserons avec le superviseur des réunions de suivi pour évaluer 
l'avancement du projet et pour discuter des problèmes rencontrés. Quant aux tâches à réaliser,
elles seront réparties entre les membres de l'équipe tout en se tenant au courant
des avancées de chacun de sorte à ce que chacun des membres puisse expliquer à lui seul l'entièreté
du projet.

Nous utiliserons également des sources d'informations externes telles que des vidéos en ligne,
des tutoriels et des forums de développeurs pour nous aider à résoudre les problèmes
rencontrés lors du développement de la plateforme, ou pour en apprendre plus sur les plateformes
actuelles de création de jeux-vidéos.

\section{Conclusion}                                % Section 6: Conclusion
\label{sec:conclusion}                             % Étiquette de la section
Le projet ZayLou Games vise à créer une plateforme de création de jeux-vidéos en 2D ou 2.5D 
simple à utiliser et accessible à un public jeune et novice. La plateforme devra être intuitive 
et permettra de scanner des cartes pour ajouter des effets dans le jeu.

Au final, ZayLou sera le rêve de tout gamer, car il permettra de faire de la
création de jeux-vidéos un jeu d'enfant. Ainsi, tout le monde, pourra donner vie à son 
imagination.

\end{document}                                      % Fin du document