\documentclass{article}             % Classe du document: article

\usepackage[french]{babel}          % Langue du document: français
\usepackage[utf8]{inputenc}         % Encodage du document: UTF-8
\usepackage[T1]{fontenc}            % Encodage des polices: T1 
\usepackage{fancybox}               % Boîtes de texte stylisées
\usepackage{hyperref}               % Liens hypertexte
\usepackage{color}                  % Couleurs
\usepackage{graphicx}               % Images

\title{Projet été 2025: ZayLou Games}             % Titre du document
\author{                            % Auteurs du document
    Lallia Diakité: 20256054 \\
    Marc Olivier Jean Paul: 20241763
}                 
\date{\today}                       % Date du document   

\begin{document}                    % Début du document

\maketitle                          % Afficher le titre

\section{Introduction}              % Section 1: Introduction
\label{sec:intro}                   % Étiquette de la section
Ce document présente le projet de l'été 2025. Il est réalisé par Lallia Diakité
et Marc Olivier Jean Paul et est supervisé par Louis-Edouard Lafontant.

Il s'agit de la présentation d'une platforme permettant de créer des jeux-
vidéos. Nous expliquerons les technologies ainsi que les mécansimes que nous y
intègrerons.

\section{Objectif}
\label{sec:objectifs}               % Section 2: Objectifs
ZayLouLou Games est une plateforme de création de mini jeux-vidéos inter-
actifs en 2D ou 2.5D. L'originalité de la plateforme se trouve dans le fait qu'elle
permet aux joueurs de scanner des cartes pour ajouter différents effets dans le
jeu. Les effets peuvent être des bonus, des malus ou même des effets surprises.

\section{Technologies envisagées}
\label{sec:technologies}       % Section 3: Technologies
Cette section présente les différentes technologies que nous envisageons d'uti-
liser pour le développement de la plateforme ZayLou Games.

\subsection{Puces NFC}
Comme mentionné plus haut, la plateforme ZayLou Games permet aux
joueurs de scanner des cartes pour ajouter des effets dans le jeu. Pour cela, nous
envisageons d'utiliser des puces NFC. NFC (Near Field Communication) est une
technologie de communication sans fil à courte portée qui permet d'échanger des
données entre deux appareils compatibles. Les puces NFC sont souvent utilisées
dans les cartes de paiement sans contact, les cartes d'identité et les étiquettes
intelligentes.

L'objectif est donc de stocker sur ces puces, grâce à des fichiers JSON, les
informations sur les effets que le joueur peut ajouter dans le jeu.

\subsection{Front-end}
React est une bibliothèque JavaScript pour la création d'interfaces utilisa-
teur. Elle est développée par Facebook et est largement utilisée pour le déve-
loppement d'applications web. Nous prévoyons donc de l'utiliser pour le déve-
loppement front-end de la plateforme ZayLou Games.

Vu que la principale fonctionnalité de la plateforme se fera sur le web, nous
avons choisi React pour sa rapidité et sa flexibilité. De plus, React permet
de créer des composants réutilisables, ce qui facilitera le développement de la
plateforme.

\subsection{Back-end}
Pour le développement back-end de la plateforme ZayLou Games, nous envi-
sageons d'utiliser Node.js. Il s'agit d'un environnement d'exécution JavaScript
côté serveur qui permet de créer des applications web rapides et évolutives.
Node.js est particulièrement adapté pour les applications en temps réel, ce
qui est idéal pour une plateforme de création de jeux-vidéos. De plus, il est facile
à intégrer avec des bases de données NoSQL comme MongoDB, ce qui facilitera
la gestion des données des utilisateurs et des jeux créés sur la plateforme.


\end{document}                      % Fin du document